\setlength{\absparsep}{18pt} % ajusta o espaçamento dos parágrafos do resumo

\begin{resumo}
\vspace{1cm}
Diariamente pessoas visitam supermercados a fim de encontrar aqueles que possam lhes proporcionar uma melhor economia em suas compras. Para tal, chegam a se deslocar para até mais de um estabelecimento no mesmo dia, tendo assim uma falsa sensação de economia, pois muitas vezes é desprezado os gastos com transportes, o que no final pode se suceder no valor que ao ser sumarizado, ultrapassa o que se esperava economizar na compra. Diante da situação, o presente trabalho propõe uma solução tecnológica para suprir tal carência, o aplicativo Superbuscado. Sua funcionalidade vai além de consultar o menor preço de uma compra, presente em aplicativos concorrentes, o aplicativo conta também com o diferencial de calcular a rota a ser percorrida pelo cliente para realizar sua compra.
O aplicativo possibilitará aos usuários saber onde encontrar o menor preço para sua compra, tendo a opção de realizá-la no supermercado mais próximo ou no que apresenta menor preço para a lista.
O aplicativo possui uma base de dados com todos os produtos e suas respectivas categorias, disponibilizados pelos supermercados associados, sendo cada estabelecimento com seus conjuntos de produtos, preços e suas coordenadas geográficas. Ao efetuar uma busca de preço, o aplicativo calcula, além do menor valor da compra em si, a distância a ser deslocada, distância essa obtida através de um recurso disponibilizado pela API do Google Maps, obtendo dados da geolocalização do usuário e das coordenadas do estabelecimento cadastrada na base de dados.
Sanando tais necessidades, torna-se possível não só economias financeiras, deveras relevante, como também a redução do tempo gasto para se realizar uma compra.

\textbf{Palavras-Chave:} Supermercado, rota, busca e preço.
\end{resumo}

% resumo em inglês
\begin{resumo}[Abstract]
\vspace{1cm}
\begin{otherlanguage*}{english}
People visit supermarkets in a daily base in order to find which one can provide them a bigger saving on their purchases. For that, they even go to more than one store on the same day, having a false sense of economy, but they often despise the transportation costs, which in the end can result in a value that when being summarized, exceed what was expected to be saved in the purchase. Hence, the present paper proposes a technological solution to supply this need, the application Superbuscado. Its functionality goes beyond the act of consulting the lowest price of a purchase, such as those apps already existing in the market, having as a differential the calculation of the route to be traveled by the customer.
The application will provide information to the users about where they can find the lowest cost to their purchase, being able to choose whether to buy at the closest supermarket or the one with the lowest price for the given shopping list.
The application has a database with all the products and their categories, made available by the associated supermarkets, each with its geographical coordinates and product prices. When performing a price search the application calculates, in addition to the smallest value of the purchase, the distance to be traveled, this distance is obtained through a feature provided by the Google Maps API, using the data of the user geolocation and the coordinates of the store registered in the database to calculate the route between the two points and its distance.
Fixing these needs would not only grant savings, which are indeed very relevant, but also reduce the time spent shopping for groceries.

\textbf{Keywords}: Supermarket, route, search and price.
\end{otherlanguage*}
\end{resumo}