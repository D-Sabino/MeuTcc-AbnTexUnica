%%%%%%%%%%%%%%%%%%%%%%%%%%%%%%%%%%%%%%%%%%%%%%%%%%%%%
%% Inserir seu texto da conclusão no campo abaixo %%
%%%%%%%%%%%%%%%%%%%%%%%%%%%%%%%%%%%%%%%%%%%%%%%%%%%%%
\chapter{Considerações Finais}

O presente trabalho cumpre ao requisito proposto, que é disponibilizar a seus usuários rotas otimizadas para realizar suas compras, sendo possível encontrar o menor preço do produto em si ou encontrar o estabelecimento mais próximo.
Desenvolver algoritmos para mapeamentos e cálculo de rotas seria bastante custoso, mas o uso da tecnologia Google Maps e suas ferramentas para este fim possibilitou tal função nesta aplicação. Possibilitando aos usuários realizarem uma lista de compras e instantaneamente descobrirem qual supermercado terá o menor valor, ou podendo levar em conta qual é o supermercado mais próximo de sua geolocalização que contém tais produtos.
De acordo com a pesquisa realizada com clientes (supostos usuários), o aplicativo seria viável e muito bem aceito por estes, pois atualmente as pessoas não dispõem de tanto tempo livre e não querem abrir mão de economizar. Na pesquisa anteriormente citada nota-se opiniões bem distintas quando se refere à aceitação do aplicativo pelos donos de comércio, pois alguns entrevistados opinaram que isso não seria tão viável para os estabelecimentos por questões de \textit{marketing}, entretanto não se pode esquecer que nos tempos atuais empresas inovadoras criam necessidades mesmo quando essas não existem, o que não é o caso desse aplicativo, já que supre uma necessidade existente. Outro ponto é o fato das empresas de sucesso, principalmente do ramo tecnológico, conquistarem os clientes para só depois conquistarem as organizações, ou seja, as empresas que querem se manter abertas se adaptam aos clientes e não o contrário.