%%%%%%%%%%%%%%%%%%%%%%%%%%%%%%%%%%%%%%%%%%%%%%%%%%%%%
%% Inserir seu texto de introdução no campo abaixo %%
%%%%%%%%%%%%%%%%%%%%%%%%%%%%%%%%%%%%%%%%%%%%%%%%%%%%%

\chapter{Introdução}

\section{Tema}
Superbuscado: Aplicativo para busca de menor preço ou rota.

\section{Problema}
Seria um aplicativo de cálculo de rota com comparação de preços para listas de compras a solução para pessoas economizarem no momento da compra de supermercado?

\section{Hipótese}
Através da aplicação é possível economizar tempo e dinheiro devido ao menor gasto de combustível ou menor custo dos produtos através do cálculo da rota que leva ao estabelecimento que contém os itens mais em conta para a lista de compras em questão.

\section{Justificativa}
Atualmente pessoas visitam diversos supermercados, visando encontrar produtos com os menores preços, perdendo tempo e dinheiro em longas viagens, que muitas vezes não cobrem o valor do desconto, em virtude disso fez-se necessário criar uma aplicação para que pessoas efetuem compras com melhor preço, baseado não só no preço dos produtos em si, mas também na melhor rota até o estabelecimento que compensará o suposto desconto em determinado supermercado.

\section{Objetivos}
	\subsection{Objetivo Geral}
	\begin{itemize}
		\item Calcular o menor preço ou a menor rota.
	\end{itemize}
	\subsection{Objetivos Específicos}
	\begin{itemize}
		\item Levar em conta o gasto de gasolina para calcular com mais precisão a eficiência de cada rota.
		\item Apresentar ao usuário de acordo com a lista de compras o cálculo da menor rota ou preço.
	\end{itemize}
	

